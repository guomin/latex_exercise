%-*- coding: UTF-8 -*-
% gougu.tex
% 勾股定理 
\documentclass[UTF8]{ctexart}

\usepackage{shapepar}
\usepackage[all, pdf]{xy}

\title{\heiti 杂谈勾股定理} 
\author{\kaishu 张三} 
\date{\today}
\bibliographystyle{plain}
\begin{document}
\maketitle 

\tableofcontents 

\newpage
\section{勾股定理在古代}

西方称勾股定理为毕达哥拉斯定理,将勾股定理的发现归功于公元前 6 世纪的毕达哥拉斯学派。该学派得到了一个法则,可以求出可排成直角三角形三边的三 元数组。毕达哥拉斯学派没有书面著作,该定理的严格表述和证明则见于欧几里 德《几何原本》的命题 47:“直角三角形斜边上的正方形等于两直角边上的两
个正方形之和。”证明是用面积做的。 

我国《周髀算经》载商高(约公元前 12 世纪)答周公问⋯⋯

$$c^2 = a^2 + b^2$$
\newpage
\section{勾股定理的近代形式} \bibliography{math}
{\songti 宋体} {\heiti 黑体} {\fangsong 仿宋} {\kaishu 楷书} 
% {\li 隶书} {\you 幼圆}

\newpage

\heartpar{%
绿草苍苍,白雾茫茫,有位佳人,在水一方。
绿草萋萋,白雾迷离,有位佳人,靠水而居。
我愿逆流而上,依偎在她身旁。无奈前有险滩,道路又远又长。 我愿顺流而下,找寻她的方向。却见依稀仿佛,她在水的中央。
我愿逆流而上,与她轻言细语。无奈前有险滩,道路曲折无已。 我愿顺流而下,找寻她的足迹。却见仿佛依稀,她在水中伫立。
}

\newpage
\xymatrix{
  a & b\ar[rd] & a+b \\
  1 &2 & 3\ar"1,1"
  \ar"1,1";"2,2"
}
\end{document}